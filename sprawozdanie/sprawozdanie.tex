\documentclass{classrep}
\usepackage[utf8]{inputenc}
\usepackage{color}
\usepackage{graphicx}

\DeclareUnicodeCharacter{00A0}{~}

\studycycle{Informatyka, studia dzienne, inż I st.}
\coursesemester{VI}

\coursename{Sztuczna inteligencja i systemy ekspertowe}
\courseyear{2019/2020}

\courseteacher{Krzysztof Lichy}
\coursegroup{poniedziałek, 10:15}

\author{
  \studentinfo{Paweł Białek}{216723} \and
  \studentinfo{Łukasz Kostrzewa}{216804}
}

\title{Zadanie 1: Piętnastka}

\begin{document}
\maketitle


\section{Cel}
{\color{black}
Celem zadania było napisanie programu rozwiązującego układankę zwaną "Piętnastką" za pomocą różnych metod przeszukiwania. Następnie należało dokonać porównania
skuteczności wszystkich analizowanych metod.}

\section{Wprowadzenie}
{\color{black}
Piętnastka jest to układanka zbudowana z ramki 4x4  i znajdujących się w niej 15 elementów oznacza to, że możliwe jest przesuwanie niektórych elementów ponieważ w ramce pozostaje puste miejsce.
\begin{figure}
\centering
\includegraphics [scale=0.15]{1024px-15-puzzle.svg}
\caption{Rysunek1. Rozwiązana układanka.}
\end{figure}
\newpage
W celu rozwiązania układanki użyto trzech metod przeszukiwania stanów:
\begin{enumerate}
\item Strategii przeszukiwania "wszerz" (BFS) - polega ona na przeszukiwaniu wszystkich wierzchołków na danym poziomie począwszy od stanu początkowego, po odwiedzeniu wszystkich wierzchołków przechodzi do następnego poziomu, aż do odnalezienia rozwiązania.
\item Strategii przeszukiwania "w głąb" (DFS) - polega ona na przeszukiwaniu wszystkich wierzchołków w danym korzeniu począwszy od stanu początkowego, po odwiedzeniu wszystkich wierzchołków przechodzi do następnego korzenia, aż do odnalezienia rozwiązania. 
\item Strategii "najpierw najlepszy": A*, z następującymi heurystykami: 
\begin{itemize}
\item metryką Hamminga - definicja
\item metryką Manhattan - definicja 
\end{itemize}
\end{enumerate}}

\section{Opis implementacji}
{\color{blue}
Należy tu zamieścić krótki i zwięzły opis zaprojektowanych klas oraz powiązań
między nimi. Powinien się tu również znaleźć diagram UML (diagram klas)
prezentujący najistotniejsze elementy stworzonej aplikacji. Należy także podać,
w jakim języku programowania została stworzona aplikacja.}

\section{Materiały i metody}
{\color{blue}
W tym miejscu należy opisać, jak przeprowadzone zostały wszystkie badania,
których wyniki i dyskusja zamieszczane są w dalszych sekcjach. Opis ten
powinien być na tyle dokładny, aby osoba czytająca go potrafiła wszystkie
przeprowadzone badania samodzielnie powtórzyć w celu zweryfikowania ich
poprawności. Przy opisie należy odwoływać się i stosować do
opisanych w sekcji drugiej wzorów i oznaczeń, a także w jasny sposób opisać
cel konkretnego testu. Najlepiej byłoby wyraźnie wyszczególnić (ponumerować)
poszczególne eksperymenty tak, aby łatwo było się do nich odwoływać dalej.}

\section{Wyniki}
{\color{blue}
W tej sekcji należy zaprezentować, dla każdego przeprowadzonego eksperymentu,
kompletny zestaw wyników w postaci tabel, wykresów (preferowane) itp. Powinny
być one tak ponazywane, aby było wiadomo, do czego się odnoszą. Wszystkie
tabele i wykresy należy oczywiście opisać (opisać co jest na osiach, w
kolumnach itd.) stosując się do przyjętych wcześniej oznaczeń. Nie należy tu
komentować i interpretować wyników, gdyż miejsce na to jest w kolejnej sekcji.
Tu również dobrze jest wprowadzić oznaczenia (tabel, wykresów), aby móc się do
nich odwoływać poniżej.}

\section{Dyskusja}
{\color{blue}
Sekcja ta powinna zawierać dokładną interpretację uzyskanych wyników
eksperymentów wraz ze szczegółowymi wnioskami z nich płynącymi. Najcenniejsze
są, rzecz jasna, wnioski o charakterze uniwersalnym, które mogą być istotne
przy innych, podobnych zadaniach. Należy również omówić i wyjaśnić wszystkie
napotkane problemy (jeśli takie były). Każdy wniosek powinien mieć poparcie we
wcześniej przeprowadzonych eksperymentach (odwołania do konkretnych wyników).
Jest to jedna z najważniejszych sekcji tego sprawozdania, gdyż prezentuje
poziom zrozumienia badanego problemu.}

\section{Wnioski}
{\color{blue}
W tej, przedostatniej, sekcji należy zamieścić podsumowanie najważniejszych
wniosków z sekcji poprzedniej. Najlepiej jest je po prostu wypunktować. Znów,
tak jak poprzednio, najistotniejsze są wnioski o charakterze uniwersalnym.}

\begin{thebibliography}{0}
  \bibitem{l2short} T. Oetiker, H. Partl, I. Hyna, E. Schlegl.
    \textsl{Nie za krótkie wprowadzenie do systemu \LaTeX2e}, 2007, dostępny
    online.
\end{thebibliography}

{\color{blue}
Na końcu należy obowiązkowo podać cytowaną w sprawozdaniu literaturę, z której
grupa korzystała w trakcie prac nad zadaniem.}

\end{document}
